A delta patch consists of differences between two images\+: the original and the destination.

Applying this delta patch to the original will produce the destination.

A delta patch uses bsdiff and bspatch tools\+: \href{http://www.daemonology.net/bsdiff/}{\tt http\+://www.\+daemonology.\+net/bsdiff/}\hypertarget{Delta_Patch_mkPatch_Presentation}{}\section{mk\+Patch Presensation}\label{Delta_Patch_mkPatch_Presentation}
The tool \hyperlink{Delta_Patch_mkPatch_tool}{mk\+Patch} is used to create a delta patch between two images. It needs an original image and a destination image to construct the delta patch between them. The images may be R\+AW binary or U\+BI volumes.

The destination image is splitted into small segments, and from each segment a patch slice is build preceeded by a delta patch header (\hyperlink{struct_delta_patch_header__t}{Delta\+Patch\+Header\+\_\+t}), describing the patch number, the offset of this patch and its length.

All patch slices are concatenated and the \char`\"{}whole patch\char`\"{} is preceeded by a delta patch meta header (\hyperlink{struct_delta_patch_meta_header__t}{Delta\+Patch\+Meta\+Header\+\_\+t}) containing the partition to patch, the U\+BI volume ID (if it concerns an U\+BI volume), the size of the segment, the the original image size and C\+R\+C32, the destination image size and C\+R\+C32, and a patch magic.

Finally the whole patch is encapsulated by a C\+WE header.

\begin{DoxyNote}{Note}
\hyperlink{Delta_Patch_mkPatch_tool}{mk\+Patch} requires bsdiff and libbz2 to be installed.
\end{DoxyNote}
\hypertarget{Delta_Patch_mkPatch_tool}{}\subsection{mk\+Patch}\label{Delta_Patch_mkPatch_tool}
This tool has the following syntax\+:

\begin{DoxyVerb}usage: mkPatch -T TARGET [-o patchname] [-S 4K|2K] [-E 256K|128K] [-N] [-v]
        {-p PART {[-U VOLID] file-orig file-dest}}

   -T, --target <TARGET>
        Specify the TARGET (mandatory - specified only one time).
   -o, <patchname>
        Specify the output name of the patch. Else use patch-<TARGET>.cwe as default.
   -S, --pagesize <4K|2K>
        Specify another page size (optional - specified only one time).
   -E, --pebsize <256K|128K>
        Specify another PEB size (optional - specified only one time).
   -N, --no-spkg-header
        Do not generate the CWE SPKG header.
   -v, --verbose
        Be verbose.
   -p, --partition <PART>
        Specify the partition where apply the patch.
   -U, --ubi <VOLID>
        Specify the UBI volume ID where apply the patch.
\end{DoxyVerb}


The --target T\+A\+R\+G\+ET is one of ar759x or ar758x respectivelly. Others targets are not supported.

The output C\+WE file is specified with -\/o. If not filled, the default name is patch-\/\+T\+A\+R\+G\+E\+T.\+cwe where T\+A\+R\+G\+ET is the T\+A\+R\+G\+ET selected by --target.

The --pagesize specify the flash device page size expected on target. It only supports 2K for 2048 page size and 4K for 4096 page size. If the --pagesize is not filled, a default value is taken according to ID specified.

The --pebsize specify the flash device physical erase block expected on target. It only supports 128K for 131072 P\+EB size and 256K for 262144 P\+EB size. If the --pebsize is not filled, a default value is taken according to ID specified.

The -\/N option requests the tool to not add a C\+WE S\+P\+KG header. This is usefull to include a delta patch C\+WE inside another C\+WE.

The -\/v requests the tool to be verbose and displays more informations.

The --partition P\+A\+RT specify which partition is concerned by this delta patch. It may one of the following\+:
\begin{DoxyItemize}
\item modem \+: The modem U\+BI image
\item tz \+: The Trust-\/\+Zone image
\item rpm \+: The Q\+CT R\+PM image
\item aboot \+: The LK bootloader image
\item boot \+: The Linux kernel image
\item system \+: The Linux rootfs image
\item lefwkro \+: The legato image
\end{DoxyItemize}

The --ubi N gives the availability to build a patch by specifying an U\+BI volume id for a static squashfs image. The squashfs image belongs to the U\+BI volume id N. N is a number from 0 up to 127. If the original and destination images are both U\+BI, this option does not need to be set.

Finally the both original and destination need to be specified.

The --partition P\+A\+RT O\+R\+IG D\+E\+ST may be specified several times, if the delta patch concerns several partitions.\hypertarget{Delta_Patch_mkPatch_Examples}{}\subsection{Examples}\label{Delta_Patch_mkPatch_Examples}
To build a delta patch between two legato squashfs images belonging to the U\+BI volume 0, do\+:

\begin{DoxyVerb}mkPatch --target ar758x -o legato-patch.cwe --partition lefwkro --ubi 0 orig/legato.squashfs dest/legato.squashfs\end{DoxyVerb}


To build a delta patch between two Linux kernels and two rootfs in the same C\+WE update package, do\+: \begin{DoxyVerb}mkPatch --target ar759x -o yocto-patch.cwe \
          --partition boot orig/boot-yocto-mdm9x40.img dest/boot-yocto-mdm9x40.img \
          --partition system orig/mdm9x40-image-minimal-swi-mdm9x40.ubi dest/mdm9x40-image-minimal-swi-mdm9x40.ubi\end{DoxyVerb}
\hypertarget{Delta_Patch_ApplyDeltaPatch}{}\section{Apply a delta patch}\label{Delta_Patch_ApplyDeltaPatch}
The delta patch C\+WE update package is applied with the tool \hyperlink{toolsTarget_fwUpdate}{fwupdate} download or with \hyperlink{le__fwupdate__interface_8h_ab68f3a7c5d4284306468e888bf6a8796}{le\+\_\+fwupdate\+\_\+\+Download} A\+PI.





Copyright (C) Sierra Wireless Inc. Use of this work is subject to license. 